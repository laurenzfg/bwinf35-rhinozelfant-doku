Die Laufzeit des Programmes wächst ungefähr Linear zu der Gesamtpixelzahl des Bildes, da der Großteil der Laufzeit für die Suche nach möglichen Hautschuppen und das Auszählen der Linienlängen genutzt wird.

Die Beschaffenheit (anzahl möglicher Hautschuppen) spielt im Average-Case eine eher geringe Rolle, da die meisten Pixel bei der Hautschuppen und Linienananlyse mit konstanter Laufzeit ausgeschlossen werden können.

Im Best-Case, also zufälligem Rauschen, in dem keine Filter zum Einstz kommen, benötigt das Programm für 6 Mio. Pixel (2000x3000) 2.3s. 
Wenn nun in einem Average-Case (Bsp. 4) einige Filterausführugen zum Finden eines im Bild vorhandenen Rhinozelfanten benötigt werden, steigt die Laufzeit marginal auf 2.6s an. 
Im Worst-Case, nämlich einem komplett einfarbigen Bild, in dem dann alle Pixel mögliche Hautschuppen sind, werden zahlreiche komplette Filterdurchläufe durchgeführt, die Laufzeit beträgt dann 30s.